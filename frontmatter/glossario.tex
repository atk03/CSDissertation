
\begin{description}
  \item[ACTIVE DIRECTORY] \hfill \\ A volte abbreviato in AD, è l'insieme di servizi di rete adottati dai sistemi operativi Microsoft e gestiti da un \emph{domain controller}. Esso si fonda sui concetti di dominio e di directory , ovvero la modalità con cui vengono assegnate agli utenti tutte le risorse della rete attraverso account utente, account computer, cartelle condivise secondo l'assegnazione da parte dell'amministratore di sistema di \emph{Group Policy}, ovvero criteri di gruppo.
  \item[API] \hfill \\ Abbreviazione di \emph{Application Programming Interface}. Indica ogni insieme di procedure disponibili al programmatore, di solito raggruppate a formare un set di strumenti specifici per l’espletamento di un determinato compito all’interno di un certo programma. La finalità è ottenere un’astrazione, di solito tra l’\emph{hardware} e il programmatore o tra \emph{software} a basso e ad alto livello semplificando così il lavoro di codifica.
  \item[ASSET] \hfill \\ Nelle tecnologie dell’informazione coincidono con il complesso di \emph{hardware}, \emph{software} e \emph{know-how} consolidato di proprietà dell’impresa.
  \item[AUDIT] \hfill \\ Attività volte a misurare la conformità di determinati sistemi, processi, prodotti a determinate caratteristiche richieste e a verificarne l'applicazione. 
  \item[BEST PRACTICE] \hfill \\ Tradotto letteralmente dall’inglese con "miglior pratica". Si intendono,con questo termine, le esperienze di successo, o comunque quelle che hanno permesso di ottenere i risultati migliori, relativamente a svariati contesti.
  \item[BOTTOM-UP] \hfill \\ Tradotto letteralmente dall’inglese con "dal basso verso l’alto". Metodo di risoluzione di un problema tramite l’identificazione prima delle sue componenti base, che vengono poi suddivise in modo tale da realizzare un sistema più complesso.
  \item[CRM] \hfill \\ Abbreviazione di \emph{Customer Relationship Management}, indica una strategia di business e di gestione dei processi, che attraverso il conseguimento dell’efficienza organizzativa permette di aumentare il fatturato aziendale garantendo al contempo un elevato livello di \emph{customer satisfaction}, incentrando il business sul cliente. 
  \item[CSV] \hfill \\ Abbreviazione di \emph{comma separated values}, è un formato di file basato su file di testo utilizzato per l'importazione ed esportazione di una tabella di dati. 
  \item[DLL] \hfill \\ Abbreviazione di \emph{Dynamic-Link Library}, è una libreria software che viene caricata dinamicamente in fase di esecuzione, invece di essere collegata staticamente a un eseguibile in fase di compilazione. 
  \item[FRAMEWORK] \hfill \\ Nello sviluppo \emph{software}, è un'architettura logica di supporto (spesso un'implementazione logica di un particolare design pattern) su cui un \emph{software} può essere progettato e realizzato, spesso facilitandone lo sviluppo da parte del programmatore.
  \item[GDPR] \hfill \\ Abbreviazione di \emph{General Data Protection Regulation}, è un regolamento dell'Unione europea in materia di trattamento dei dati personali e di privacy. 
  \item[GUI] Abbreviazione di \emph{Graphical User Interface}, è un tipo di interfaccia utente che consente all’utente di interagire con la macchina controllando oggetti grafici convenzionali.
 \item[HELP DESK] \hfill \\ Servizio destinato a fornire supporto, all'utente o al cliente, relativamente a prodotti o servizi informatici ed elettronici allo scopo di risolvere problemi o fornire indicazioni su prodotti come computer, apparecchiature elettroniche o \emph{software}. 
 \item[IIS] \hfill \\ Abbreviazione di \emph{Internet Information Services}, è un complesso di servizi server Internet per sistemi operativi Microsoft.
 \item[IT GOVERNANCE] \hfill \\ Insieme di linee guida volte alla gestione dei rischi informatici e all'allineamento dei sistemi alle finalità dell'attività.
 \item[IT LIFE CICLE MANAGEMENT] \hfill \\ Insieme di processi, procedure, competenze e strumenti utilizzati nella gestione dei \emph{Configuration Item} che compongono il sistema informativo in tutte le fasi del loro ciclo di vita.
\item[ITIL]\hfill \\ Abbreviazione di \emph{Information Technology Infrastructure Library}, è l'insieme di linee guida nella gestione dei servizi IT. Consiste in una serie di pubblicazioni che forniscono indicazioni sull'erogazione di servizi IT di qualità e sui processi e mezzi necessari a supportarli da parte di un'organizzazione. 
\item[ITSM] \hfill \\ Abbreviazione di \emph{IT service management}, è la disciplina che si occupa di pianificare, progettare e gestire i sistemi IT di un'organizzazione.
\item[MSDN] \hfill \\ Abbreviazione di \emph{ Microsoft Developer Network}, è la divisione di Microsoft incaricata di mantenere i rapporti con gli sviluppatori e gli amministratori di sistema. È responsabile di mantenere la maggior parte della documentazione riguardante i prodotti Microsoft.
\item[NUGET] \hfill \\ Strumento di gestione dei pacchetti per le piattaforme di sviluppo Microsoft.  
\item[OPEN-CLOSE PRINCIPLE] \hfill \\ Secondo principio del modello di sviluppo SOLID, essoafferma che le entità (classi, moduli, funzioni, ecc.) software dovrebbero essere aperte all'estensione, ma chiuse alle modifiche; in maniera tale che un'entità possa permettere che il suo comportamento sia modificato senza alterare il suo codice sorgente.
\item[SAM] \hfill \\ Abbreviazione di \emph{software Asset Management}, si riferisce ad una pratica di business che comprende la gestione e l’ottimizzazione
dell’acquisto, distribuzione, mantenimento, utilizzo e smaltimento di prodotti \emph{software}. Il SAM è l’intera infrastruttura e l’insieme di
processi necessari per l’efficacia della gestione, controllo e protezione degli asset
\emph{software} attraverso tutti gli stadi del loro ciclo di vita. 
\item[SDK] \hfill \\ Abbreviazione di \emph{Software Development Kit}, indica un insieme di strumenti per lo sviluppo e la documentazione di \emph{software}. \item[SISTEMA CLIENT/SERVER] \hfill \\  Architettura di rete nella quale genericamente un computer (client) si connette ad un server per la fruizione di un certo servizio appoggiandosi alla sottostante architettura protocollare.
\item[SERVICE PROVIDER] \hfill \\ A volte abbreviato in SP, indica imprese che forniscono servizi di vario tipo e appartenenti a settori differenti. 
\item[SINGLE RESPONSABILITY PRINCIPLE] \hfill \\ Primo principio del modello di sviluppo SOLID, esso afferma che ogni elemento di un programma (classe, metodo, variabile) deve avere una sola responsabilità, e che tale responsabilità debba essere interamente incapsulata dall'elemento stesso. Tutti i servizi offerti dall'elemento dovrebbero essere strettamente allineati a tale responsabilità. 
\item[SPLA] \hfill \\ Abbreviazione di \emph{Service Provider License Agreement}, indica una tipologia di licenze mirate alle organizzazioni che desiderano offrire servizi \emph{software} ai loro clienti. La differenza tra una licenza SPLA e una normale è che mentre la seconda può essere utilizzata solamente da chi la ha acquistata, la prima permette al compratore di renderla disponibile a terzi. Questo permette ad alcune organizzazioni di offrire servizi basati su \emph{software} Microsoft
gestendo autonomamente le licenze, in maniera trasparente agli utilizzatori finali.
\item[STAGE-IT] \hfill \\ Iniziativa che mira a mettere in contatto imprese e studenti. Offre un punto di contatto comune dove gli studenti possono scegliere tra diverse offerte di
stage delle varie aziende e le aziende possono entrare in contatto con gli studenti in maniera diretta.
\item[TICKET] \hfill \\ Richiesta di assistenza, tracciata e presa in carico da un servizio di assistenza tecnica.
\item[WEB SERVICE] \hfill \\ Sistema \emph{software}
progettato per supportare l’interoperabilità tra diversi elaborati su di una medesima rete in un contesto distribuito. Questo permette di offrire diversi tipi di servizi tramite il \emph{web}.
\end{description}
